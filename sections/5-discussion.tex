\section{Discussion}

% - Global evolution

Initially developed as an internal project, Onyxia has gained recognition beyond the realm of Insee or the French administration. Convinced of the potential of cloud technologies to foster autonomy and leverage the full potential of data science, several organizations now have a production instance of Onyxia running, and multiple others are in the process of either testing or implementing one. Besides, the choice of Onyxia as the reference data science platform in the context of the AIML4OS project should further facilitate its adoption within the ESS. This trend is naturally very beneficial to the Onyxia project, as it moves from a project developed in open-source - but mainly at Insee - to a full open-souce project with a growing base of contributors. This in turn facilitates its adoption by other organizations, since it gives more guarantees on its sustainability independently from Insee's strategy. The governance of the project is currently evolving to reflect this trend. For instance with the organization of monthly community calls and the creation of a public channel and roadmap for the project\footnote{All information are available on the GitHub depository of the project : \url{https://github.com/InseeFrLab/onyxia}}.

Despite this success, we observe several limitations to the widespread adoption of the project in organizations. First, it is essential to remind that the fundamental choice made by organizations that adopt Onyxia is not the software itself, but the underlying technologies: containerization (through Kubernetes) and object storage. These technologies can represent substantial entry costs for organizations, as they demand a significant commitment to developing and maintaining skills which are not readily found in NSOs. Yet, the general trend towards cloud-native solutions among data-centric organizations suggests a favorable shift that could mitigate these challenges over time.

Similarly, the transition towards cloud-native technologies induces entry costs for statisticians. First, they often deal with a loss of references regarding where computations actually happen: while they may be accustomed to performing computation on centralized servers rather than a personal computer, the container adds a layer of abstraction that make the location hard to grasp at first. But the major perceived change in this paradigm is the loss of data persistence. In traditional setups - either a personal computer or a server accessed through a virtual desktop - the code, the data and the computing environment are kind of mixed in a black box fashion. On the contrary, containers have no persistence by design. While object storage provides this persistence, a proper use of these infrastructures for statistical projects require a variety of tools and corresponding skills: using a version control system for the code, interacting with the object storage API to store the data, providing configuration files or secrets as inputs, etc. In a way, these entry costs can be seen as the "price" of autonomy: thanks to cloud-native technologies, statisticians now have access to scalable and flexible environments that enable them to experiment more freely, but this autonomy requires a significant skills upgrade which may be overwhelming at first and limit adoption. However, our experience at Insee suggests that this effect can largely be mitigated through a combination of training statisticians to development best practices and accompanying statistical projects when transitioning to cloud infrastructures.

While Onyxia has significantly democratized access to cloud-native technologies for statisticians, the journey towards integrating data science within NSOs encompasses broader organizational challenges beyond the technical realm. In particular, the deployment of our first machine learning model in production highlighted the necessity of overcoming skill compartmentalization across IT, business, and innovation teams. Effective collaboration and the seamless integration of machine learning into statistical processes require not only shared goals and methodologies but also a convergence of diverse skills and knowledge. Addressing these challenges involves strategic measures such as embedding data science capabilities within business teams for better alignment with project objectives, and initiating cross-disciplinary training to harmonize the toolsets and languages across teams. Ultimately, the successful transition to a data science-driven approach in statistical production is contingent upon a balanced strategy that marries technical solutions like Onyxia with comprehensive organizational adjustments, fostering a culture of collaboration, continuous learning, and innovation.
