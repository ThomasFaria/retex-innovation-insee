\section{Discussion}

Initially developed as an internal project, Onyxia has gained recognition beyond the realm of Insee or the French administration. Aware of the necessity to foster autonomy in order to leverage the full potential of data science, several organizations now have a production instance of Onyxa running, and multiple others are in the process of either testing or implementing one. Besides, the choice of Onyxia as the reference data science platform in the context of the One-Stop-Shop for Artificial Intelligence/Machine Learning for Official Statistics (AIML4OS) will further facilitate its adoption within the ESS. This trend is naturally very beneficial to the Onyxia project, as it moves from a project developed in open-source - but mainly at Insee - to a full open-souce project with a growing base of contributors. This in turn facilitates its adoption by other organizations, since it gives more guarantees on its sustainability independently from Insee's strategy. The governance of the project is currently evolving to reflect this trend, for instance with the organization of monthly community calls and the creation of a public channel and roadmap for the project\footnote{All information are available on the GitHub depository of the project : \url{https://github.com/InseeFrLab/onyxia}}.

Despite this success, we observe several limitations to the widespread adoption of the projects in organizations. Again, we want to point out that the fundamental choice made by organizations that adopt Onyxia is not the software itself, but the underlying technologies: containerization (through Kubernetes) and object storage. These technologies can represent substantial entry costs for organizations, as they demand a significant commitment to developing and maintaining competencies which are not readily found in NSOs, and also require changes to the organizational structures. Yet, the general trend towards cloud-native solutions among data-centric organizations suggests a favorable shift that could mitigate these challenges over time.

For statisticians, the transition to cloud-native development environments introduces its own set of challenges. such as adapting to non-persistent environments and working within containers, which may initially disorient those accustomed to more traditional setups. However, targeted training and support, such as the Insee training schools and mentorship programs, aim to bridge these gaps by promoting best practices and easing the transition to modern, reproducible, and collaborative workflows.

\begin{itemize}
    \item Cout d'entrée important pour le statisticien :
    \begin{itemize}
        \item Non-persistence de l'environnement $\rightarrow$ git + stockage objet
        \item Travail dans un conteneur $\rightarrow$ perte de repères sur l'environnement
        \item Mais formation : bonnes pratiques + écoles de formation Insee + accompagnements
    \end{itemize}
    \item SSP Cloud :
    \begin{itemize}
        \item Instance ouverte $\rightarrow$ absence de données sensibles $\rightarrow$ grosse limitation des cas d'usage réalisables + frustrations $\rightarrow$ en résumé, difficile de maximiser à la fois innovation et sécurité (pb sur-contraint)
        \item $\rightarrow$ résolution via le choix de l'innovation max car sujet des échanges inter-administration de données complexe + le SSP Cloud a pavé la voie à des instances internes, plus fermées $\rightarrow$ stratégie assumée "platform-as-a-package" : projet open-source packagé $\rightarrow$ facilité ++ de réinstanciation
        \item Pas une plateforme de diffusion de données $\rightarrow$ pas de stratégie globale de gouvernance $\rightarrow$ le sujet de la méta-donnée n'est pas abordé.
    \end{itemize}
    \item Gouvernance :
    \begin{itemize}
        \item Quelle organisation ? Equipe DS centralisée qui vient en appui ou data scientists dans les orgas métiers ? Collaboration avec les équipes infos ? (cf. graphique orga/compétences de Romain)
    \end{itemize}
\end{itemize}


\subsubsection{Governance and collaboration challenges in ML Projects}

During the deployment of our first model into production, we encountered several governance challenges that we had not anticipated. As explained in Section \ref{sec:mlops}, we initiated the project with three distinct teams: an IT team comprised of developers, the core business team responsible for maintaining the Sirene directory, and finally, the innovation team consisting of data scientists and data engineers. The creation of the SSP Lab\footnote{Name of the innovation team.} at Insee in 2018 was justified to support the business teams on projects with relatively precise objectives in terms of statistical production and to experimentally introduce new methods into the statistical process, including machine learning methods. Initially, the innovation team's sole objective was to conduct an experiment and let the business team judge its relevance based on the results obtained. However, due to the urgency and quality of the results, the innovation team continued to collaborate with the business team to assist in production deployment.

This phase highlighted several issues, with the primary one being the compartmentalization of skills. The innovation team had limited knowledge of business issues and was unable to make certain decisions independently. The business team, on the other hand, had very few data science skills and was unable to manage the production deployment of the experiment alone. Finally, the IT team, although well-versed in DevOps best practices, had no knowledge of MLOps approach and of the methodological tools used in machine learning. Additionally, the programming languages used by the production and innovation teams were different, which have slowed down the model's production deployment. Before using MLflow and its Model Registry functionality, the preprocessing done in Python had to be reprogrammed in Java, which proved to be very tedious. We realized how code duplication was a source of errors and needed to be avoided as much as possible.

To address these challenges, several measures were implemented at Insee. Firstly, a data scientist was hired in September 2023 and integrated into the business team to take responsibility for the model in production, its monitoring, and retraining. The goal is to have someone who fully understands the business issues to quickly integrate new features as needed. Furthermore, to anticipate the arrival of multiple machine learning models in production, a Python training plan for developers was launched to enhance their skills with open-source software and align development and production languages.


parler de limportance de la communication avec les gestionnaire pour la confiance
