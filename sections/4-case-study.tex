
Faire un petit paragraphe pour présenter ce que le chapitre va évoquer. Premiere mise en prod défi et challenges et techniques utilisées. Objectif est d'être le plus concret possible 

\subsection{Introduction et contexte à l'insee au niveau métier}


Un petit topo sur le contexte à l'Insee codification faite avec sicore et manuellement très couteuse et pas stimulante
Possible car équipe au pied du mur $\rightarrow$ Innovation possible mais pas voulue très contrainte en terme de timing

Faire un rappel méthodo de ce qu'on cherche à faire

\subsection{Démarrage du projet comme les projets expérimental et prise en compte des contraintes}

On a un passé avec beaucoup d'expérimentation mais pas vraiment de mise en prod. Un des rares projets où des le début l'enjeu de la mise en production a été prise en compte. concilier problématiques
informatiques et métiers.

ici on explique quel modèle on a utilisé et pourquoi fasttexte => java etccc

1er question, ou on peut travailler ? 
\begin{itemize}

    \item projet ML plusieurs tâches : modularité de l'infra + collaboration (git indispensable, stockage partagé)
    \item Illustration de la diversité des taches nécessaires dans un projet de ML et modularité indispensable de l'infra utilisée (reprendre infra Big Data trop spécifique et onyxia cool)
    \item dans notre cas données ouverte donc possibilité d'utiliser le ssp cloud
    \item Rappeler les contraintes/prérequis que cela impose : utilisation de Git n'est pas aisée et nécessite des formations (mise en place d'un cursus de formateurs pour former à l'Insee), sauvegarde des données sur MinIO et pas en  local car environnement éphémère
\end{itemize}

2eme question, comment travailler ?
\begin{itemize}
    \item Choix de langage de développement : python. Dire débat R et python, Insee est passé à du tout R mais ecosystème ML plutot python. Ne pas opposer les deux, ils sont complémentaire gnagna
    \item On travaille sur des notebook en local on obtient des bon résultats mais on arrive rarement à les mettre à l'echelle. 
    \item Rappeler tous les défauts des notebook pour la mise en prod. 
\end{itemize}


rappeler les nouveaux enjeux pour les projets de ML (model versionning, logging parameters) 
L'utilisation du ssp cloud permet d'accéder à plusieurs logiciels tous interconnectés pour favoriser le developpement de projet de machine learning favorisant une approche MLOps
Objectif d'appliquer cette approche durant ce projet.

\subsection{MLflow as the cornerstone of the project}

Logiciel qui permet de suivre cette approche = MLflow et c'est dispo sur ssp cloud

\begin{itemize}
    \item Why Mlflow ?
    \item Projects
    \item Models
    \item Tracking server
    \item Model registry
\end{itemize}

\subsection{Embracing the power of Onyxia from training to deployment}

\begin{itemize}
    \item Distributing trainings with Argo workflows
    \item Deployment on the kubernetes cluster (freed from DSI) with fastAPI  $\rightarrow$ conteneurisation Docker
    \item Automatiser les déploiements avec argoCD
\end{itemize}

Environnement dev et production très proche $\rightarrow$ passage en prod facilité
\begin{itemize}
    \item Transmission d'une image
    \item Transmission d'une API
\end{itemize}

\subsection{Monitoring of the model}

\begin{itemize}
    \item Enjeu du monitoring => indispensable
    \item data drift/ concept drift
    \item Pour APE : Création d'un dashboard (faire un super graphs qui récap tout)
    \item encore on utilise les trucs du datalab (argocd pour le déploiement, argoworkflow pour les cronjob quotidien)
\end{itemize}

\subsection{Annotation en continue}

\begin{itemize}
    \item Evaluer la performance en créant un fichier test golden standard -> intégré au dashboard
    \item Amélioration du jeu d'entrainement en corrigeant les erreurs
    \item passage en NAF2025 très bientôt gros enjeu
    \item tout ca réalisé sur le datalab avec LabelStudio
    \item Rappeler les problèmes rencontrés (faire comprendre aux équipes métiers que c'est ultra important pour améliorer la performance, nécessite ressources humaines importantes..)
\end{itemize}


\subsection{Gouvernance d'un projet de ML/ challenges}

