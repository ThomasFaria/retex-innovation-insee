\section{Onyxia: an open source project to build cloud-native data science platforms}

The Onyxia project, initiated by the French Public Service and available at
onyxia.sh, is an open source project aimed at creating self-sufficient data science environments in the cloud or on-premises. This project can be seen as a “Platform as a Package” (PaaP) solution for organisations wishing to create a data science environment based on cloud technologies.

\subsection{Making cloud-technologies accessible to statisticians}

Our technology watch and literature review highlighted cloud-native technologies, in particular containerization and object storage, as instrumental in building a data science platform that is both scalable and flexible. Building on these insights, we established our initial on-premise Kubernetes cluster in 2020, integrating it with MinIO, an open-source object storage system designed to work seamlessly with Kubernetes. Yet, our first experiments highlighted a significant barrier to their widespread adoption: the complexity of their integration. This is an important consideration when building data architectures that prioritize modularity — an essential feature for the flexibility we aim to achieve. For instance, due to MinIO's compatibility with the Amazon S3 API, the storage source could easily be switched without requiring substantial modifications to one managed by a public cloud provider. However, modularity of the architecture components also entails that any data application launched on the cluster must be configured so as to communicate with all the components. For instance, in a big data setup, configuring Spark to operate on Kubernetes while interacting with datasets stored in MinIO requires an intricate set of configurations (specifying endpoints, access tokens, etc.), a skill set that typically lies beyond the expertise of statisticians.

This very insight is really the base of the Onyxia project : choosing technologies that foster autonomy won't actually foster autonomy if their complexity acts as a deterrent from widespread adoption in the structure. In recent years, statisticians at Insee already need to adapt to a changing environment in termes of their everyday tools : transitioning from proprietary software (SAS) to open-source ones (R, Python), acculturating to technologies that improve reproducibility (version control with Git), consuming and developing APIs, etc. These changes, that make their activity more and more akin to the one of software developers, already imply significant training and changes in the modalities of work everyday. In this regard, adoption of cloud-technologies was utterly dependent on making them readily accessible.

% TODO: figure Onyxia components

To bridge this gap, we developed Onyxia, an application that essentially acts as interface between the modular components that compose the architecture. The main entrypoint of the user is a user-friendly web application\footnote{\url{https://github.com/InseeFrLab/onyxia-ui}} that enables users to launch services from a data science catalog (see section \ref{ssec:catalog}) as running containers on the underlying Kubernetes cluster. The interface between the UI and Kubernetes is done by a lightweight custom API\footnote{\url{https://github.com/InseeFrLab/onyxia-api}}, that essentially transforms the application request of the user into a set of manifests to deploy Kubernetes resources. For a given application, these resources are packaged under the form of Helm charts, a popular way of packaging potentially complex applications on Kubernetes \cite{gokhale2021creating}. Although users can configure a service to tailor it to their needs, they will most of the time just launch a service out-of-the-box and be able to start developing. This point really illustrates the added value of Onyxia in facilitating the adoption of cloud technologies. By injecting authentication information and configuration into the containers at the initialization, we ensure that users can launch and manage data science services in which they can interact seamlessly with the data from their bucket on MinIO, their sensitive information (tokens, passwords) stored in Vault, etc. This automatic injection, coupled with the pre-configuration of data science environments in Onyxia's catalogs of images\footnote{\url{https://github.com/InseeFrLab/images-datascience}} and associated helm-charts\footnote{\url{https://github.com/InseeFrLab/helm-charts-interactive-services}}, make it possible for users to execute complex workloads - such as running distributed computations with Spark on Kubernetes using data stored in S3, or training deep-learning models on a GPU - without getting bogged down by the technicalities of configuration.

\subsection{Architecture principles aimed at fostering autonomy}

- no vendor-lockin (enfermement de la structure $\rightarrow$ coût (licences) et des pratiques $\rightarrow$ fige les compétences)

Another significant risk with commercial cloud services is vendor lock-in
(Opara-Martins, Sahandi, and Tian 2016). Organisations may become increas-
ingly integrated with the tools and services of one cloud provider, potentially
making transitions to other platforms difficult and resource-intensive. This de-
pendency can restrict adaptability and might increase operational costs over
time. Recognizing these challenges, there is a trend towards endorsing cloud-
neutral strategies (Opara-Martins, Sahandi, and Tian 2017). These methods
aim to engage with cloud services in a way that reduces reliance on a single
vendor’s specific solutions.

- cloud-native : onyxia n'est pas le choix fondamental, le parti pris est sur le choix sous-jacent : conteneurisation + stockage objet
- users have access to the underlying Kubernetes cluster -> full autonomy to launch services tailored to their needs
- Acculturation aux bonnes pratiques par l'usage

\label{ssec:catalog}
\subsection{An extensive catalogue of services to cover the entire lifecycle of data science projects}

Axe : principes
- A catalog of services which covers the entire lifecycle of a data science project
- production-ready : outils d'automatisation (-> autonomie)

\subsection{Building commons : an open-source project and an open-innovation platform}

- Orientation plateforme : instance vivante d'Onyxia, ouverte, collaborative, sandbox (cf. ref papier SSP Cloud sur l'aspect plateforme)
- Innovation ouverte $\rightarrow$ littérature
- Open-data
- Instance de partage : formations reproductibles + utilisation dans les écoles de stats + hackathons (organisation annuelle du funathon cf. one-stop-shop)
