\subsection{Onyxia}

Axe : mise à dispo des technos cloud $\rightarrow$ favoriser l'autonomie.
\begin{itemize}
    \item Convergence des choix d'archi. Mais suffisant pour garantir l'autonomie : non $\rightarrow$ les outils de l'éco-système s'adressent plutôt à des informaticiens (ex : difficulté de configurer Spark sur du stockage objet en mode kube)
    \item Eco système découplé, mais exigeant $\rightarrow$ compétences diverses.
    \item Enjeu : faciliter l'acces aux ressources cloud pour les statisticiens (qui doit déjà s'acculturer à la reproductibilité $\rightarrow$ convergence avec les outils des développeurs) $\rightarrow$ double décalage qui demande une assistance
    \item IHM Onyxia comme liant technique
\end{itemize}

Axe : principes
\begin{itemize}
    \item production-ready : outils d'automatisation (-> autonomie)
    \item no vendor-lockin (enfermement de la structure $\rightarrow$ coût (licences) et des pratiques $\rightarrow$ fige les compétences)
    \item cloud-native : onyxia n'est pas le choix fondamental, le parti pris est sur le choix sous-jacent : conteneurisation + stockage objet
\end{itemize}

\subsection{SSP Cloud}

\begin{itemize}
    \item Orientation plateforme : instance vivante d'Onyxia, ouverte, collaborative, sandbox (cf. ref papier SSP Cloud sur l'aspect plateforme)
    \item Innovation ouverte $\rightarrow$ littérature
    \item Open-data
    \item Instance de partage : formations reproductibles + utilisation dans les écoles de stats + hackathons (organisation annuelle du funathon cf. one-stop-shop)
    \item A catalog of services which covers the entire lifecycle of a data science project
    \item Acculturation aux bonnes pratiques par l'usage
\end{itemize}
