\section{Onyxia: an open source project to build cloud-native data science platforms}

The Onyxia project, initiated by the French Public Service and available at
onyxia.sh, is an open source project aimed at creating self-sufficient data science environments in the cloud or on-premises. This project can be seen as a “Platform as a Package” (PaaP) solution for organisations wishing to create a data science environment based on cloud technologies.

\subsection{Making cloud-technologies accessible to statisticians}

Our technology watch and literature review highlighted cloud-native technologies, in particular containerization and object storage, as instrumental in building a data science platform that is both scalable and flexible. Building on these insights, we established our initial on-premise Kubernetes cluster in 2020, integrating it with MinIO, an open-source object storage system designed to work seamlessly with Kubernetes. Yet, our first experiments highlighted a significant barrier to their widespread adoption: the complexity of their integration. This is an important consideration when building data architectures that prioritize modularity — an essential feature for the flexibility we aim to achieve. For instance, due to MinIO's compatibility with the Amazon S3 API, the storage source could easily be switched without requiring substantial modifications to one managed by a public cloud provider. However, modularity of the architecture components also entails that any data application launched on the cluster must be configured so as to communicate with all the components. For instance, in a big data setup, configuring Spark to operate on Kubernetes while interacting with datasets stored in MinIO requires an intricate set of configurations (specifying endpoints, access tokens, etc.), a skill set that typically lies beyond the expertise of statisticians.

This very insight is really the base of the Onyxia project : choosing technologies that foster autonomy won't actually foster autonomy if their complexity acts as a deterrent from widespread adoption in the structure. In recent years, statisticians at Insee already need to adapt to a changing environment in termes of their everyday tools : transitioning from proprietary software (SAS) to open-source ones (R, Python), acculturating to technologies that improve reproducibility (version control with Git), consuming and developing APIs, etc. These changes, that make their activity more and more akin to the one of software developers, already imply significant training and changes in the modalities of work everyday. In this regard, adoption of cloud-technologies was utterly dependent on making them readily accessible.

% TODO: figure Onyxia components

To bridge this gap, we developed Onyxia, an application that essentially acts as interface between the modular components that compose the architecture. The main entrypoint of the user is a user-friendly web application\footnote{\url{https://github.com/InseeFrLab/onyxia-ui}} that enables users to launch services from a data science catalog (see section \ref{ssec:catalog}) as running containers on the underlying Kubernetes cluster. The interface between the UI and Kubernetes is done by a lightweight custom API\footnote{\url{https://github.com/InseeFrLab/onyxia-api}}, that essentially transforms the application request of the user into a set of manifests to deploy Kubernetes resources. For a given application, these resources are packaged under the form of Helm charts, a popular way of packaging potentially complex applications on Kubernetes \cite{gokhale2021creating}. Although users can configure a service to tailor it to their needs, they will most of the time just launch a service out-of-the-box and be able to start developing. This point really illustrates the added value of Onyxia in facilitating the adoption of cloud technologies. By injecting authentication information and configuration into the containers at the initialization, we ensure that users can launch and manage data science services in which they can interact seamlessly with the data from their bucket on MinIO, their sensitive information (tokens, passwords) stored in Vault, etc. This automatic injection, coupled with the pre-configuration of data science environments in Onyxia's catalogs of images\footnote{\url{https://github.com/InseeFrLab/images-datascience}} and associated helm-charts\footnote{\url{https://github.com/InseeFrLab/helm-charts-interactive-services}}, make it possible for users to execute potentially complex workloads - such as running distributed computations with Spark on Kubernetes using data stored in S3, or training deep-learning models using a GPU - without getting bogged down by the technicalities of configuration.

% TODO: figure : from launching a service to its deployment on Kube + S3, Vault ?

\subsection{Architectural choices aimed at fostering autonomy}

The Onyxia project is based on a few structuring principles, with a central theme : fostering autonomy. First, at the level of the organization by preventing vendor lock-in. In order to get a competitive edge, many commercial cloud providers develop applications and protocols that customers need to use to access cloud resources but that are not interoperable, greatly complexifying potential migrations to another cloud platform \cite{opara2016critical}. Recognizing these challenges, there is a trend towards endorsing cloud-neutral strategies \cite{opara2017holistic} in order to reduce reliance on a single vendor’s specific solutions. In contrast, the use of Onyxia is inherently not restrictive: when an organization chooses to use it, it chooses the underlying technologies - containerization and object storage - but not the solution. The platform can be deployed on any Kubernetes cluster, either on-premise or in public clouds. Similarly, although Onyxia was designed to be used with MinIO because it is an open-source object-storage solution, but is also compatible with objects storage solutions from various cloud providers (AWS, GCP).

The other important level at which Onyxia fosters autonomy is at the level of users. Proprietary softwares that have been used intensively in official statistics - such as SAS or STATA - also produce a vendor lock-in phenomenon. The costs of licensing are high and can evolve quickly, and users are tied in certain ways of performing computations, preventing progressive upskilling. On the contrary, Onyxia aspires to be removable; we want to enhance users' familiarity and comfort with the underlying cloud technologies rather than act as a permanent fixture in their workflow. An illustrative example of this philosophy is the platform's approach to user actions: for tasks performed through the UI, such as launching a service or managing data, we provide users with the equivalent terminal commands, promoting a deeper understanding of what actually happens on the infrastructure when triggering something. Furthermore, all the services offered through Onyxia's catalog are open-source.

% TODO: figure show code

Naturally, the way Onyxia makes statisticians more autonomous in their work depends on their needs and familiarity with IT skills. Statisticians that just want to have access to extensive computational resources to experiment with new data sources or statistical methods will have access in a few clicks to easy-to-use, pre-configured data science environments, so that they can directly start to work and prototype their solution. However, many users want to go deeper and build actual prototypes of production applications for their projects: configuring initialization scripts to tailor the environments to their needs, deploying an interactive app that delivers data visualisation to users of their choice, deploying other services than those available in our catalogs, etc. For these advanced users to continue to push the boundaries of innovation, Onyxia gives them access to the underlying Kubernetes cluster. This means that users can freely open a terminal on an interactive service and interacts with the cluster - within the boundaries of their namespace - in order to apply custom resources and deploy custom applications or services.

Besides autonomy and scalability, the architectural choices of Onyxia also foster reproducibility of statistical computations. In the paradigm of containers, the user must learn to deal with resources which are by nature ephemeral, since they only exist at the time of their actual mobilization. This fosters the adoption of development best practices, notably the separation of the code — put on an internal or open-source forge such as GitLab or GitHub — the data — persisted on a specific storage solution, such as MinIO — and the computing environment. While this requires an entry cost for users, it also helps them to conceive their projects as pipelines, i.e. a series of sequential steps with well-defined inputs and outputs (akin to directed acyclic graph (DAG)). The projects developed in that manner are usually more reproducible and portable — they can work seamlessly on different computing environments — and thus also more readily shareable with peers.

% TODO: figure separation processes + DAG

\label{ssec:catalog}
\subsection{An extensive catalogue of services to cover the entire lifecycle of data science projects}

Axe : principes
- A catalog of services which covers the entire lifecycle of a data science project
- production-ready : outils d'automatisation (-> autonomie)

% TODO: figure catalogue onyxia

\subsection{Building commons : an open-source project and an open-innovation platform}

- Utilisation de communs : briques open-source
- Orientation plateforme : instance vivante d'Onyxia, ouverte, collaborative, sandbox (cf. ref papier SSP Cloud sur l'aspect plateforme)
- Innovation ouverte $\rightarrow$ littérature
- Open-data
- Instance de partage : formations reproductibles + utilisation dans les écoles de stats + hackathons (organisation annuelle du funathon cf. one-stop-shop)
