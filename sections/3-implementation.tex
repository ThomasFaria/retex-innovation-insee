\section{Onyxia: an open source project to build cloud-native data science platforms}



The Onyxia project, initiated by the French Public Service and available at
onyxia.sh, is an open source project aimed at creating self-sufficient data science environments in the cloud or on-premises. This project can be seen as a “Platform as a Package” (PaaP) solution for organisations wishing to create a data science environment based on cloud technologies.

\subsection{Making cloud-technologies accessible to statisticians}

Our technology watch and literature review highlighted cloud-native technologies, in particular containerization and object storage, as instrumental in building a data science platform that is both scalable and flexible. Building on these insights, we established our initial Kubernetes cluster in 2020, integrating it with MinIO, an open-source object storage system designed to work seamlessly with Kubernetes. Yet, our first experiments highlighted a significant barrier to their widespread adoption by statisticians: the complexity of their integration. This is an important insight when building data architectures in a modular fashion. By definition, this modularity lies at the source of the flexibility we are looking for - for instance, MinIO being compatible with the Amazon S3 API, the storage source could almost effortlessly be switched to a public cloud provider. However, it also means that any data application launched on the cluster must be configured so as to communicate with all the required components. For instance, in a big data setup, configuring Spark to operate on Kubernetes and interact with datasets stored in MinIO requires an intricate set of configurations (specifying endpoints, access tokens, etc.), a skill set that typically lies beyond the expertise of statisticians.



Axe : mise à dispo des technos cloud $\rightarrow$ favoriser l'autonomie.
- convergence des choix d'archi. Mais suffisant pour garantir l'autonomie : non $\rightarrow$ les outils de l'éco-système s'adressent plutôt à des informaticiens (ex : difficulté de configurer Spark sur du stockage objet en mode kube)
- Eco système découplé, mais exigeant $\rightarrow$ compétences diverses.
- Enjeu : faciliter l'acces aux ressources cloud pour les statisticiens (qui doit déjà s'acculturer à la reproductibilité $\rightarrow$ convergence avec les outils des développeurs) $\rightarrow$ double décalage qui demande une assistance
- IHM Onyxia comme liant technique

\subsection{Architecture principles aimed at fostering autonomy}

- no vendor-lockin (enfermement de la structure $\rightarrow$ coût (licences) et des pratiques $\rightarrow$ fige les compétences)

Another significant risk with commercial cloud services is vendor lock-in
(Opara-Martins, Sahandi, and Tian 2016). Organisations may become increas-
ingly integrated with the tools and services of one cloud provider, potentially
making transitions to other platforms difficult and resource-intensive. This de-
pendency can restrict adaptability and might increase operational costs over
time. Recognizing these challenges, there is a trend towards endorsing cloud-
neutral strategies (Opara-Martins, Sahandi, and Tian 2017). These methods
aim to engage with cloud services in a way that reduces reliance on a single
vendor’s specific solutions.

- cloud-native : onyxia n'est pas le choix fondamental, le parti pris est sur le choix sous-jacent : conteneurisation + stockage objet
- Acculturation aux bonnes pratiques par l'usage

\subsection{An extensive catalogue of services to cover the entire lifecycle of data science projects}

Axe : principes
- A catalog of services which covers the entire lifecycle of a data science project
- production-ready : outils d'automatisation (-> autonomie)

\subsection{Building commons : an open-source project and an open-innovation platform}

- Orientation plateforme : instance vivante d'Onyxia, ouverte, collaborative, sandbox (cf. ref papier SSP Cloud sur l'aspect plateforme)
- Innovation ouverte $\rightarrow$ littérature
- Open-data
- Instance de partage : formations reproductibles + utilisation dans les écoles de stats + hackathons (organisation annuelle du funathon cf. one-stop-shop)
