\section{Introduction}
\label{sec:introduction}

In recent years, the European Statistical System (ESS) has committed to leverage non-traditional data sources in order to improve the process of statistical production, an evolution that is encapsulated by the concept of Trusted Smart Statistics \cite{ricciato2019trusted}. This dynamic is accompanied by innovations in the statistical processes, so as to be able to take advantage of the great potential of these new sources (greater timeliness, increased spatio-temporal resolution, etc.), but also to cope with their complexity or imperfections. At the forefront of these innovations are machine-learning methods and their promising uses in the coding and classification fields, data editing and imputation \cite{gjaltema2022high}. The multiple challenges faced by statistical institutes because of this evolution are addressed in the Bucharest Memorandum on Official Statistics in a Datafied Society (Trusted Smart Statistics), which predicts that "the variety of new data sources, computational paradigms and tools will require amendments to the statistical business architecture, processes, production models, IT infrastructures, methodological and quality frameworks, and the corresponding governance structures", and consequently invites the ESS to assess the required adaptations and prioritize them \cite{bucharest2018}.

In line with these recommendations, much work has been done in the context of successive projects at the European level in order to operationalize the use of non-traditional data sources in the production of official statistics. Within the scope of the ESSnet Big Data II project (2018-2020), National Statistical Offices (NSOs) have been working across a wide range of themes (online job vacancies, smart energy, tracking ships, etc.) in order to put together the building blocks for using these sources in actual production processes and identify their limitations \cite{essnetbigdata2}. However, while a substantial amount of work has been devoted to developing methodological frameworks \cite{descy2019towards, salgado2020mobile}, quality guidelines \cite{kowarik2022quality} as well as devising business architectures that make third-party data acquisition more secure \cite{ricciato2018processing}, not much has been said about the IT infrastructures and skills needed to properly deal with these new objects.

The characteristics of big data sources make them particularly complex to process, be it their volume, their velocity (speed of creation or renewal) or their variety (structured but also unstructured data, such as text and images). Besides, the "skills and competencies to automate, analyse, and optimize such complex systems are often not part of the traditional skill set of most National Statistical Offices" \cite{ashofteh2021data}. Not incidentally, an increasing number of public statisticians trained as data scientists have joined NSOs in recent years. Within its multiple meanings, the term “data scientist” reflects the increased involvement of statisticians in the IT development and orchestration of their data processing operations, beyond merely the design or validation phases \cite{davenport2012data}. However, based on our observations at Insee and other French statistical offices, the ability of these new data professionals to derive value from big data sources and machine learning methods is limited by several challenges.

A first challenge is related to the lack of proper IT infrastructures to tackle the new data sources that NSOs now have access to as well as the accompanying need for new statistical methods. For instance, big data sources require huge storage capacities and often rely on distributed computing frameworks to be processed \cite{liu2013computing}. Similarly, the adoption of new statistical methods based on machine learning algorithms often requires IT capacities — in particular, GPUs (graphical processing units) — to massively parallelize computations \cite{saiyeda2017cloud}. Such resources are not readily available in traditional IT infrastructures. Furthermore, these new infrastructures generally require specific skills — especially to build and maintain them — that are not easily found in NSOs.

Another major challenge lies in equipping statisticians with development environments that enable them to experiment more freely. The essence of innovation in statistical work lies in the ability to swiftly adapt to and incorporate new tools and methodologies. This agility is hampered when statisticians depend excessively on IT departments to provision resources or install new software packages. In traditional setups — personal computers or virtual desktops on centralized architectures — IT departments generally prioritize security and system stability over the provision of new services, which limits the innovation potential. Besides, these rigid environments make it harder to implement development best practices, such as collaborative work — which requires environments where experiments can be easily shared with peers — and reproducibility.

A third challenge is related to the difficulty of transitioning from innovative experiments to production-ready solutions. Even when statisticians have access to development environments in which they can readily experiment, the step towards deploying an application or a model is generally very large. Production environments often differ from development environments in such a way that the additional development costs are needed to go from a proof of concept to an industrialized solution that actually serves users can limit the feasibility of this transition. Furthermore, in the case of machine learning projects, models that have been deployed require a proper monitoring to ensure that they maintain their accuracy and utility over time, and generally require periodic or continuous improvements. Again, this pleads for more flexible environments that enable statisticians to manage the complete lifecycle of their data science projects in a more continuous way.

We argue that these various challenges have an underlying common theme: the need for more autonomy. The ability of data science methods to improve and potentially transform the production of official statistics crucially depends on the ability of statisticians to carry out innovative experiments more freely. To do so, they need to have access to substantial and diverse computing resources that enable them to tackle the volume and diversity of big data sources and leverage machine learning methods to better deal with these data. Such experimental projects require, in turn, flexible development environments that foster collaborative work in order to capitalize the diversity of profiles and skills that compose project teams. Finally, to derive value from these experiments, statisticians require tools to deploy applications as proof-of-concepts and orchestrate their statistical operations autonomously.

Against this background, we developed Onyxia: an open source project that enables organizations to deploy data science platforms that foster innovation by giving statisticians more autonomy\footnote{\url{https://github.com/InseeFrLab/onyxia}}. This paper aims at describing the full thought process that led to this project and at exemplifying how it empowers statisticians at Insee, thus becoming a cornerstone of our innovation strategy. Section~\ref{sec:principles} provides an in-depth analysis of the data ecosystem's latest developments, casting light on the technological choices that have shaped the development of a modern data science environment tailored to the specific needs of statisticians. In particular, we show how cloud-native technologies — particularly containers and object storage — are key to building scalable and flexible environments that can enhance autonomy while promoting reproducibility in the production of official statistics. However, despite their appealing attributes for modern data science applications, the complexity of configuring and utilizing cloud technologies often poses barriers to their broad adoption. In section~\ref{sec:implementation}, we detail the core of the Onyxia project: how we made cloud technologies accessible to statisticians through a user-friendly interface and an extensive catalogue of ready-to-use data science environments, while circumventing potential vendor lock-in effects for both the institution and their users. We also show how providing an open-innovation instance of Onyxia, the SSP Cloud, greatly facilitated the adoption of these technologies and fostered improved development practices. Finally, through the case study of the classification of French companies' activity, section~\ref{sec:mlops} illustrates how leveraging these technologies greatly facilitated the deployment of machine learning models at Insee in alignment with the industry best practices — namely, MLOps principles.
