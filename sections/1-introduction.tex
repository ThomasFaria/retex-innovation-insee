In recent years, the European Statistical System (ESS) has committed to leverage the potential offered
by new data sources and statistical methods. The Scheveningen Memorandum on Big Data and Official Statistics
acknowledges the opportunities and challenges of Big Data for official statistics,
noting in particular that "developing the necessary capabilities and skills to effectively explore
Big Data is essential for their integration into the European Statistical System" \cite{scheveningen2013}.
More recently, the Bucharest Memorandum on Official Statistics in a Datafied Society (Trusted Smart Statistics)
\cite{bucharest2018} further states that "the variety of new data sources, computational paradigms and tools
will require amendments to the statistical business architecture, processes, production models, IT
infrastructures, methodological and quality frameworks, and the corresponding governance structures" in order
to integrate big data into the regular production of official statistics.

These principles involve simultaneously the need for new technical skills as well as innovative IT solutions. Not incidentally, an increasing number of public statisticians trained as data scientists have joined NSIs in recent years. However, these new profiles often find themselves isolated in national statistical systems, and their ability to deliver value is limited by several challenges.